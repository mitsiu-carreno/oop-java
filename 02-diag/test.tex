\documentclass[10pt]{article}
% pdflatex <input>
\usepackage{listings}
\usepackage{xcolor}
\usepackage{geometry}  % Page layout
\usepackage{fancyhdr}  % Header and footer management
\usepackage{parskip}   % Space between paragraphs

\lstset{
  language=Python,
  backgroundcolor=\color{white},   % White background
  basicstyle=\ttfamily\footnotesize, % Typewriter style and smaller font size
  keywordstyle=\color{black},        % Black for keywords like def, if, etc.
  commentstyle=\color{gray},         % Gray for comments
  stringstyle=\color{darkgray},      % Dark gray for strings
  numbers=left,                      % Line numbers on the left
  stepnumber=1,                      % Number every line
  numberstyle=\tiny\color{gray},     % Small and gray numbers
  frame=single,                      % Frame around code
  captionpos=b,                      % Position of caption (at the bottom)
  breaklines=true,                   % Line breaking
  breakatwhitespace=true,            % Break lines at spaces
  showspaces=false,                  % Do not show space characters
  showstringspaces=false             % Do not show spaces inside strings
}
% Page layout settings
\geometry{a4paper, margin=1in}

\begin{document}

\section*{Evaluación diagnóstica Python}
\textmd{Mitsiu Alejandro Carreño Sarabia}

% Student Information
\noindent
\textbf{Nombre completo:} \underline{\hspace{10cm}} \\
\textbf{Mátricula:} \underline{\hspace{10cm}} \\

% Instructions
\noindent \textbf{Indicaciones:}
\normalfont{La siguiente evaluación tiene como propósito conocer el nivel de familiaridad del alumno con el lenguaje Python.} \\
\textbf{Esta evaluación no tiene valor sobre la calificación.}
\vspace{0.3cm}

\textmd{1. Para qué sirve el siguiente comando }
\begin{lstlisting}
python --version
\end{lstlisting}

\vspace{2cm} % Space for the answer

\textmd{2. Cuál es el error (lógico) en el siguiente código}
\begin{lstlisting}
if 5 > 2:
print("Five is greater than two!")
\end{lstlisting}

\vspace{4cm} % Space for the answer

\textmd{3. ¿La siguiente línea es válida en Python? de ser así ¿para qué sirve?}
\begin{lstlisting}
#print("Hello World")
\end{lstlisting}

\vspace{2cm} % Space for the answer

\textmd{4. ¿La siguiente línea es válida en Python? de ser así ¿qué operación está realizando?}
\begin{lstlisting}
x, y, z = "Orange", "Banana", "Cherry"
\end{lstlisting}

\vspace{3cm} % Space for the answer

\textmd{5. ¿Qué código se debe agregar en la línea 3 para ejecutar la función my\_function?}
\begin{lstlisting}
def my_function():
  print("Hello from a function")

\end{lstlisting}

\vspace{2cm} % Space for the answer

\textmd{6. Para cada una de las siguientes variables escribe a la derecha su tipo de dato}
\begin{lstlisting}
x = "Hello World"
x = 20
x = 20.5
x = 1j
x = ["apple", "banana", "cherry"]
x = ("apple", "banana", "cherry")
x = range(6)
x = {"name" : "John", "age" : 36}
x = {"apple", "banana", "cherry"}
x = True
x = bytearray(5)
x = None
\end{lstlisting}

\textmd{7. ¿Escribe cuál es el código para conocer el tipo de dato de una variable x?}
\vspace{3cm} % Space for the answer

\textmd{8. ¿El siguiente código es válido? Justifique su respuesta}
\begin{lstlisting}
x = 5
y = "John"
print(x + y)
\end{lstlisting}

\vspace{3cm} % Space for the answer

\textmd{9. ¿El siguiente código es válido? Justifique su respuesta}
\begin{lstlisting}
def my_function(country = "Norway"):
  print("I am from " + country)
my_function()
\end{lstlisting}

\vspace{4cm} % Space for the answer

\textmd{10. ¿Cuál es el resultado del siguiente código?}
\begin{lstlisting}
def my_function(x):
  return 5 * x

print(my_function(3))
\end{lstlisting}

\vspace{2cm} % Space for the answer

\textmd{11. ¿Qué hace que un profesor sea excelente?}
\vspace{4cm} % Space for the answer
\end{document}
